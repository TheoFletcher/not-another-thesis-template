% The thesis class has all the usual options that 'book' has
% Extra options:
% - singlespacing, onehalfspacing, doublespacing = line spacing
\documentclass{thesis}

% Setup title page
\title{Title Thesis}
\subtitle{Optional subtitle}
\author{Author Name}
\department{Department}
\institution{University of Somewhere}
\documenttype{\forthedegreeof{Doctor of Philosophy}}
%\date{}  % Defaults to \today

% Optionally include a crest or logo on the title page
\renewcommand{\maketitlecrest}{%
    \includegraphics[height=4cm]{example_crest} 
}
% The example crest was produced by Ali Muksin,
% and is used under the Free License from vecteezy.com
% https://www.vecteezy.com/vector-art/7944555-education-logo-design-vector-template


% These commands control different formatting options
% To modify the formatting, redefine them with:
%   \renewcommand{\commandname}{...}

% - \headingfont    Formatting for all headings
% - \partfont
% - \chapterfont
% - \sectionfont
% - \subsectionfont
% - \subsubsectionfont
% - \paragraphfont
% - \subparagraphfont

% - \quotefont      Formatting of quote, quotation, etc...
% - \quotespacing   Line spacing for quote, etc...


% The thesis class does NOT import any bibliography packages,
% this must be done manually
% Adjust the package options accordingly
\usepackage[backend=biber,
            sorting=none,
            hyperref=true,
            citestyle=chem-acs,
            bibstyle=nature]{biblatex}
\renewcommand{\cite}[1]{\supercite{#1}}
\renewcommand*{\bibsetup}{\singlespacing}
%\addbibresource{refences}


% Used to generate filler text - delete this when using the template
\usepackage{lipsum}


\begin{document}

\frontmatter

 % Different formatting of the title page is possible with, e.g:
 % {\sffamily \maketitle \cleardoublepage}
\maketitle \cleardoublepage

\begin{abstract}
    \lipsum[1]
\end{abstract}

\begin{dedication}
    Thank you to the author of this thesis template
\end{dedication}

% \begin[Custom Name]{acknowledgements} to set 'Custom Name' as the title
\begin{acknowledgements}
    \lipsum[2]
\end{acknowledgements}

\tableofcontents
\listoffigures
\listoftables

\chapter{Introduction}
\lipsum[3-8]


\mainmatter

\chapter{The First Chapter} \label{chap:first_chapter}
\lipsum[9-11]
\section{Section}
\lipsum[12]

% The thesis class should have loaded the following packages:
% amsmath, amsfonts, amssymb, physics
\begin{equation} \label{eq:an_equation}
    a^2 = \vec{x} \cdot \vec{y} .
\end{equation}

\lipsum[13]
\subsection{Subsection}
\lipsum[14]
\subsubsection{Subsubsection}
\lipsum[15]
\paragraph{Paragraph} 
\lipsum[16][1-5]
\subparagraph{Subparagraph} 
\lipsum[17][1-5]


\chapter{All About Referencing}

The thesis class loads the package \texttt{cleveref}, which makes referencing easy. For example, the previous chapter was \cref{chap:first_chapter}, which included \eqref{eq:an_equation}. \Cref{chap:first_chapter} can be used for references at the start of a sentence.


\chapter{}

The template also works if no chapter name is provided.


% If you want the backmatter BEFORE any appendices, include this:
{
\backmatter
\chapter{Conclusions}
\lipsum[22-26]
\printbibliography[heading=bibintoc,title={References}]
}
% This disables chapter numbering for the backmatter, 
% but enables it again for the appendicees


% Optional appendix
\appendix
\chapter{The First Appendix}
\lipsum[18]
\section{Section of the appendix}
\lipsum[19-21]

% If you want the backmatter AFTER any appendices, include this:
%\backmatter
%\chapter{Conclusions}
%\lipsum[22-26]
%\printbibliography[heading=bibintoc,title={References}]

\end{document}
